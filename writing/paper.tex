\documentclass{article}

% these packages let you do math
\usepackage{amsmath}
\usepackage{amssymb}

% we need these packages for fancy R tables
\usepackage{booktabs}
\usepackage{float}
\usepackage{colortbl}
\usepackage{xcolor}

% these packages play with the spacing/margins of the document. Uncomment the commands on lines 16 and 17 to see what they do.
\usepackage{a4wide}
\usepackage{setspace}
\usepackage{geometry}
\usepackage{parskip}
%\doublespacing
%\geometry{margin=1.5in}

% this package helps us with including images. Setting the graphics path makes it easier to refer to things in the \includegraphics command.
\usepackage{graphicx}
\graphicspath{ {../figures/} }

% make some hyperlinks using the \href command
\usepackage{hyperref}
\hypersetup{
    colorlinks=true,
    linkcolor=black,
    urlcolor=blue
}

% set the author, title, and date of the document. \maketitle adds it to the document.
\author{Peter (Tun-Shuo) Lee }
\title{Analysis of 2002 Incarceration data from NLSY97}
\date{Sping 2022}

\begin{document}
\maketitle

\section{Introduction}

Systematic inequity between racial groups in the US has always been a widely debated topic. Differences in employment, education, income, have placed 
many families in a disadvantage that spans across generations. Amongst the racial treatment discussions, perhaps one of most contentious is of the criminal 
justice system and whether their representatives judge the cases differently based on the defendants' ethnicity. This paper will analyze the racial composition 
of the \texttt{National Longitudinal Study of Youth}'s data from 2002, in particular, the racial and gender distribution of the incarcerated. While this short analysis aims not to provide
a definitive conclusion to the ongoing debate, it does provide quantitative evidence on the difference (or lack there of) of incarcerations in youth across race and gender.  

\newpage

\section{Analysis}

The data set NLSY97 included 8621 individuals' gender and race information, as well as whether they were incarcerated during each month in 2002. 
By first summing the number of months, and dividing by the total number of individuals within each gender and race group, we essentially find the 
rate of incarceration of the youths for each race and gender. The data is summarized in the table below: 

\input{../tables/incarcerations_by_racegender.tex}

An interesting trend arises in that males have a generally higher incarceration rate than females. This trend is present for all races, except for Mixed Race NonHispanic, where there are no recorded 
male incarcerations. Our findings can be better interpretted once visualized: 

\begin{figure}[H]
    \begin{center}
        \includegraphics[width=.85\textwidth]{incarcerations_by_racegender.png}
    \end{center}
    \caption{Mean Number of Incarcerations in 2002 by Race and Gender (this is the LaTeX caption, not the ggplot title)}
    \label{fig:graph}
\end{figure}

For Black, Hispanic and Non-Black Non-Hispanic individuals, the incarceration rate for male is higher than female. The difference is the largest for Black males, then Hispanic, and the lowest being NonHispanic, NonBlack. 
With both male and female considered, Black individuals are the most likely to be incarcerated. Hispanic teens comes next, and lastly, NonBlack/ NonHispanic. Surprisingly, the incarceration rate for Mixed Raced, NonHispanic women
are higher than for women of all other races, and yet, there are no recorded incarcerations for Mixed Race, NonHispanic men. 

\section{Regression}

I regressed the incarceration rate on categorical variables representing race and gender. I omitted Black Females to avoid multicolinearity. 

$$
    incarceration rate = \beta_{0} + \beta_{1}Hispanic + \beta_{2}Mixed Race (NonHispanic) + \beta_{3} NonBlack/ NonHispanic + \beta_{4} Male + \varepsilon
$$

The results are as follows: 

\input{../tables/regress_incarcerations_by_racegender.tex}

\end{document}